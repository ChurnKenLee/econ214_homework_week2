\documentclass[12pt]{article}
\usepackage[final]{graphicx}
\usepackage{rotating}
\usepackage{setspace}
\usepackage{amsmath}
\usepackage{amssymb}
\usepackage{amsthm}
\usepackage{breqn}
\usepackage[utf8]{inputenc}
\usepackage[style=authoryear, backend=biber]{biblatex}

\addbibresource{econ214_homework_week2_bibliography.bib}
\doublespace
\title{Econ 214 reading homework for \textcite{baxter_king_1993} and \textcite{woodford_2011}}
\author{Churn Ken Lee}
\date{}
\begin{document}
\maketitle

\section{\cite{baxter_king_1993}}
\begin{enumerate}
    \item No effect.
    \item Permanent increase in government spending has a larger impact. 
    This is due to the investment amplification channel. 
    With a permanent increase in government spending, desired capital stock is higher in the long run.
    In order to accumulate capital, households supply more labor.
    With a temporary increase in government spending, desired capital stock is the same in the long run.
    Hence instead of accumulating capital, households draw down capital stuck to finance consumption in the short run.
    \item Increase in goverment expenditure financed via lump-sum taxation increases labor supply, which decreases the marginal product of labor for a given level of capital stock.
    \item Increased government expenditure financed via lump-sum taxation increases output more. This is because distortionary taxes reduce households' marginal benefit of increasing labor supply and capital stock, dampening their response to a reduction in full income.
    \item Yes.
    It is modeled as public capital that increases the productivity of private capital and labor.
    \item Basic government purchases do not directly affect the productivity of labor and capital unlike spending on public capital.
    The effect of the former on output is driven by households increasing labor suuply in response to reductions in full income. This in turn raises the marginal product of capital, driving up investment and capital stock.
    
\end{enumerate}

\section{\cite{woodford_2011}}

\printbibliography
\end{document}